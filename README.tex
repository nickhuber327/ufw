% Created 2022-12-01 Thu 12:53
% Intended LaTeX compiler: pdflatex
\documentclass[11pt]{article}
\usepackage[utf8]{inputenc}
\usepackage[T1]{fontenc}
\usepackage{graphicx}
\usepackage{longtable}
\usepackage{wrapfig}
\usepackage{rotating}
\usepackage[normalem]{ulem}
\usepackage{amsmath}
\usepackage{amssymb}
\usepackage{capt-of}
\usepackage{hyperref}
\author{Nicholas Huber}
\date{\today}
\title{ufw-refresh.ros}
\hypersetup{
 pdfauthor={Nicholas Huber},
 pdftitle={ufw-refresh.ros},
 pdfkeywords={},
 pdfsubject={},
 pdfcreator={Emacs 28.2 (Org mode 9.6)}, 
 pdflang={English}}
\begin{document}

\maketitle
\tableofcontents


\section{Refresh the IPs in the UFW by hostnames}
\label{sec:orgdbe2b8e}
\subsection{Roswell Header stuff}
\label{sec:org5baac55}
\begin{verbatim}
#!/bin/sh
#|-*- mode:lisp -*-|#
#|
exec ros -Q -- $0 "$@"                  ; ; ; ; ; ; ; ; ; ;
|#

\end{verbatim}

\subsection{Info Block}
\label{sec:org779ba16}
\begin{verbatim}
;;;; UFW refresh
;;;  The purpose of this script is to check refresh which hosts are allowed to access a machine
;;;  by checking their current IPs, using hostnames, and checking them against IPs currently allowed access
;;;  in UFW.
;;;  Script must be run as ROOT user

;;; *TODO* modify to allow for arbitrary number of hosts
;;; *TODO* remove host specific functions
\end{verbatim}

\subsection{Roswell init stuff}
\label{sec:orge205fee}
voodoo magic:

\begin{verbatim}
(progn ;;init forms
  (ros:ensure-asdf)
  #+quicklisp(ql:quickload '() :silent t)
  )

(defpackage :ros.script.ufw-refresh.3876675733
  (:use :cl))
(in-package :ros.script.ufw-refresh.3876675733)
\end{verbatim}

\subsection{Hostnames and ports}
\label{sec:orgb0aca76}
Lists of the hostnames and their ports to check:
\begin{verbatim}
(defparameter *hostname* "jumpserver")
(defparameter *port* "22")
\end{verbatim}

\subsection{Get Current IPs}
\label{sec:orga274716}
Functions for getting the current IPs based off of hostname
\texttt{get-current-ip} takes hostname and calls \texttt{GETENT AHOSTS} and parses the result to get its current ip

\begin{verbatim}
(defun get-current-ip (hostname)
 "Get the current IP of a host using the hostname by calling GETENT"
  (car (uiop:split-string (uiop:run-program (uiop:strcat "/usr/bin/getent ahosts " hostname) :output :string))))

(defun get-jumpserver-ip ()
 "Get the current IP of the JUMPSERVER"
  (get-current-ip *hostname*))

\end{verbatim}

\subsection{Get Old IPs}
\label{sec:org7bb209a}
Functions for grabbing the IPs currently in UFW's rules
\texttt{get-old-ip} calls \texttt{UFW STATUS} to get the IPs currently in the rules 
\begin{verbatim}
(defun get-old-ip ()
 "Get the old IP of a host by taking the last element of a list of the output of UFW\'s status"
  (cdr (uiop:split-string (uiop:run-program "/usr/sbin/ufw status"))))

\end{verbatim}

\subsection{Delete Old Rules}
\label{sec:orgdacd34c}
Functions to delete old UFW rules
\texttt{delete-old-rule} takes an IP and a Port and calls UFW with \texttt{UFW DLETE} to remove that rule based on the specific IP and Port

\begin{verbatim}
(defun delete-old-rule (ip port)
 "Delete the old rule for an allowing an IP"
  (uiop:run-program (uiop:strcat "/usr/sbin/ufw delete allow from " ip " to any port " port) :output :string))

(defun delete-old-jumpserver ()
 "Delete the old rule allowing JUMPSERVER access"
  (delete-old-rule (get-jumpserver-ip) *port*))

\end{verbatim}

\subsection{Define New Rules}
\label{sec:orgf1f679b}
Functions used to define new UFW rules
\texttt{add-new-rule} takes an IP and Port and calles \texttt{UFW INSERT} to insert a new rule allowing that IP on that Port

\begin{verbatim}
(defun add-new-rule (ip port)
 "Add a new rule allowing current IP access"
  (uiop:run-program (uiop:strcat"/usr/sbin/ufw insert 1 allow from " ip " to any port " port) :output :string))

(defun add-new-jumpserver ()
 "Add a new rule allowing JUMPSERVER\'s new IP access"
  (add-new-rule (get-jumpserver-ip) *port*))

\end{verbatim}

\subsection{Replace IPs}
\label{sec:org0c2e722}
Functions used to delete old UFW rules and define new ones
\texttt{replace-ip} takes the old IP and new IP and calls the respective functions to delete and add new rules
\textbf{TODO}: generalize \texttt{replace-ip}
\textbf{TODO}: possbily replace rules regardless of whether or not they match?

\begin{verbatim}
(defun replace-ip (new-ip old-ip)
 "Replace old IP by calling DELETE-OLD-JUMPSERVER and ADD-NEW-JUMPSERVER"
  (cond ((string= new-ip old-ip)
         (progn
           (delete-old-jumpserver)
           (add-new-jumpserver)))))

(defun replace-jumpserver ()
 "Replace JUMPSERVER\'s IP by calling REPLACE-IP with GET-OLD-IP and GET-JUMPSERVER-IP as arguments"
  (replace-ip (get-old-ip) (get-jumpserver-ip)))

\end{verbatim}

\subsection{Main Function}
\label{sec:org2e68c83}
Main function to run program

\begin{verbatim}
(defun main (&rest argv)
  (declare (ignorable argv))
  (replace-jumpserver))

\end{verbatim}

\subsection{Roswell Footer Info}
\label{sec:org992dc2a}
\begin{verbatim}
;;; vim: set ft=lisp lisp:
\end{verbatim}
\end{document}